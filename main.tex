% \documentclass[sigconf, authordraft]{acmart}
\documentclass[table, xcdraw, sigconf,review, anonymous]{acmart}
\acmConference[ESEC/FSE 2018]{12th Joint Meeting of the European Software Engineering Conference and the ACM SIGSOFT Symposium on the Foundations of Software Engineering}{4--9 November, 2018}{Lake Buena Vista, Florida}

\usepackage{booktabs} % For formal tables
\usepackage{url}

% Copyright
%\setcopyright{none}
%\setcopyright{acmcopyright}
%\setcopyright{acmlicensed}
\setcopyright{rightsretained}
%\setcopyright{usgov}
%\setcopyright{usgovmixed}
%\setcopyright{cagov}
%\setcopyright{cagovmixed}

% DOI
\acmDOI{XX.YY/ZZ}

% ISBN
\acmISBN{ZZ-YY-24-ZZ/QQ/A}

%Conference
\acmConference[FSE'18]{Florida}{Nov 2018}{} 
\acmYear{2018}
\copyrightyear{2018}

\acmPrice{15.00}

\acmSubmissionID{123-A12-B3}

\begin{document}
\title{Research and Teaching
Tools for Data-Driven Search-Based Software Engineering}
% \titlenote{Produces the permission block, and
%   copyright information}
% \subtitle{Extended Abstract}
% \subtitlenote{The full version of the author's guide is available as
%   \texttt{acmart.pdf} document}

\author{Vivek Nair, everyone else, Tim Menzies}
% \authornote{Dr.~Trovato insisted his name be first.}
% \orcid{1234-5678-9012}
\affiliation{%
  \institution{North Carolina State University, USA}
%   \streetaddress{P.O. Box 1212}
  \city{Raleigh} 
  \state{NC} 
  \postcode{27606}
}
\email{dchen20@ncsu.edu,  wfu@ncsu.edu,  tim@menzies.us}


\begin{abstract}
This paper provides a sample of a \LaTeX\ document which conforms,
somewhat loosely, to the formatting guidelines for
ACM SIG Proceedings.\footnote{This is an abstract footnote}
\end{abstract}

%
% The code below should be generated by the tool at
% http://dl.acm.org/ccs.cfm
% Please copy and paste the code instead of the example below.
%
% \begin{CCSXML}
% <ccs2012>
%  <concept>
%   <concept_id>10010520.10010553.10010562</concept_id>
%   <concept_desc>Computer systems organization~Embedded systems</concept_desc>
%   <concept_significance>500</concept_significance>
%  </concept>
%  <concept>
%   <concept_id>10010520.10010575.10010755</concept_id>
%   <concept_desc>Computer systems organization~Redundancy</concept_desc>
%   <concept_significance>300</concept_significance>
%  </concept>
%  <concept>
%   <concept_id>10010520.10010553.10010554</concept_id>
%   <concept_desc>Computer systems organization~Robotics</concept_desc>
%   <concept_significance>100</concept_significance>
%  </concept>
%  <concept>
%   <concept_id>10003033.10003083.10003095</concept_id>
%   <concept_desc>Networks~Network reliability</concept_desc>
%   <concept_significance>100</concept_significance>
%  </concept>
% </ccs2012>
% \end{CCSXML}

% \ccsdesc[500]{Computer systems organization~Embedded systems}
% \ccsdesc[300]{Computer systems organization~Redundancy}
% \ccsdesc{Computer systems organization~Robotics}
% \ccsdesc[100]{Networks~Network reliability}


\keywords{ACM proceedings, \LaTeX, text tagging}


\maketitle

% \input{samplebody-conf}
\section{Introduction}

    \subsection{Why Actionable Software Analytics?}
        Software analytics aim to
        obtain actionable insights
        from software artifacts that
        help practitioners accomplish tasks 
        related to software development, systems, 
        and users.
        
        The key part of software analytics is to provide 
        actionable insights, which means it have to 
        ignore the unimportant and irrelevant information, 
        while provide the most important one clearly and directly.
        For example, if the manager is driving toward a cliff, 
        we don’t want to distract her with analytics 
        telling her about the clouds in the sky or 
        the flowers on the side of the road. Instead, 
        we want our smart analytics to shout in her ear, 
        “There’s a cliff up ahead! Turn left immediately!”
        
        So why actionable software analytics? 
        For software vendors, managers, developers and users, 
        the actionable insights provided by software analytics 
        are considered as the core deliverable~\cite{tan2016defining}. 
        In fact, for software analytics with no actionable advice provided, 
        it is hard to imagine its usefulness and existence. 
        More importantly,
        actionable insight is the key driver for businesses 
        to invest in data analytics initiatives~\cite{sawyer2013bi}. 

    \subsection{Why heuristics work?}
        Heuristics, together with logic and probability 
        are three central ideas in the intellectual history of our mind.
        Unlike statistical optimization procedures, 
        heuristics do not try to find the best solution
        but rather find a good-enough solution). 
        Because of that, 
        people tends to believe heuristics produce second-best results, 
        while optimization is always better.
        In spite of this one, Gigerenzer pointed out the other five common but erroneous beliefs about heuristics\cite{gigerenzer2008heuristics}, e.g. "People rely on heuristics only in routine
        decisions of little importance", 
        "People with higher cognitive capacities employ
        complex weighting and integration of
        information; those with lesser capacities use
        simple heuristics", and 
        "More information and computation is always
        better".
        
        In the late 90s, Gigerenzer discovered 
        the take-the-best heuristic is
        both more frugal and more accurate than 
        multiple regression
        
        Czerlinski et al. confirmed 
        the take-the-best heuristic is
        both more frugal and more accurate than 
        multiple regression result for 20 real-world problems\cite{czerlinski1999good}. 
        In 2006, Brighton showed for the first time that 
        the take-the-best heuristic is 
        often more accurate and frugal than
        complex nonlinear algorithms, including neural networks, 
        exemplar models, and classification and regression trees~\cite{brighton2006robust}. 
        
        So why do heuristics work? As Gigerenzer summarized in 2008, heuristics exploit evolved capacities that
        come for free, and thus they can provide solutions to problems
        that are different from strategies of logic and probability. 
        
        The take-the-best heuristic is designed to help choose
        between two alternatives, thus could be adjusted for 
        classification tasks, where each object needs to be assigned to
        one of several classes. 
        Basically, the resulting classification rule is a 
        fast and frugal tree, which allows for a quick decision 
        at each node of the tree~\cite{martignon2003naive}. 
        
    \subsection{What is Fast and Frugal Trees(FFTs)? }
        Fast-and-frugal trees were proposed by Martignon et al. at 2018
        as decision trees with exactly two branches extending
        from each node, where either one or both branches is an exit
        branch leading to a leaf~\city{martignon2008categorization}. 
        That is to say, in an FFT,  every question posed by a node will 
        trigger an immediate decision. Because FFTs have an exit
        branch on every node, they usually make decisions faster
        than common decision trees while
        simultaneously being easier to understand and use.
    
         provides an example of fast and frugal tree. 
        
        ~\ref{fig:fft-demo} presents an FFT designed to classify a change
        as being bug or not.
        The three nodes in the FFT correspond to the results of
        three features/cues: wmc, rdc, ce
        
        % thal indicates the result of a thallium
        % scintigraphy, a nuclear imaging test that shows how well
        % blood flows into the heart while exercising or at rest. The
        % result of the test can either be normal (n), indicate a fixed
        % defect (fd), or a reversible defect (rd). The second node
        % uses the cue cp, indicating a patient’s type of chest pain,
        % which can be either typical angina (ta), atypical angina (aa),
        % non-anginal pain (np), or asymptomatic (a). Finally, ca
        % indicates the number of major vessels colored by fluoroscopy,
        % a continuous x-ray imaging tool, whose values can range
        % from 0 to 3.
        
        
        \begin{figure}[!t]
            \centering
            \includegraphics[width=3.1in]{img/fft-demo.png}
            \caption{An example Fast Frugal Tree on the log4j dataset. 
            (Generated by 
            \href{https://econpsychbasel.shinyapps.io/ShinyFFTrees/}{ShinyFFTrees} 
            from Nathaniel.)
            }
            \label{fig:fft-demo}
        \end{figure}
    
    \subsection{Why use FFTs?}
        A lot of recent works have shown that the fast and frugal tree 
        could performs comparably well to the complex models in different kinds of domains, e.g. public health, medical risk management, performance science, etc.~\cite{jenny2013simple, laskey2014comparing, raab2015power}


\section{Methods}
    \subsection{Data}
    
    \subsection{Models}
    
    \subsection{Evaluation}
    

\section{Results}

\section{Discussions}

\section{Conclusions}
 

% \bibliographystyle{ACM-Reference-Format}
% \bibliography{sample-bibliography}

\bibliographystyle{ACM-Reference-Format}
\bibliography{References} 

\end{document}
